\documentclass[10pt]{article}

\usepackage{geometry}
 \geometry{
 a4paper,
 margin=0.5in}
\usepackage{mathtools}
\usepackage{amsmath}
\usepackage{multicol}
\usepackage{wrapfig}
\usepackage{tcolorbox}
\newcommand{\High}{\mathrm{High}}
\newcommand{\Low}{\mathrm{Low}}
\newcommand{\NKeys}{\mathrm{NKeys}}
\newcommand{\NPages}{\mathrm{NPages}}
\newcommand{\NTuples}{\mathrm{NTuples}}
\newcommand{\ProbeCost}{\mathrm{ProbeCost}}
\newcommand{\Cost}{\mathrm{Cost}}
\usepackage{minted}
\usemintedstyle{vs}

\setlength{\parindent}{0em}
\setlength{\parskip}{0.4em}
\DeclarePairedDelimiter{\ceil}{\lceil}{\rceil}
\newtcolorbox{mybox}{colback=red!5!white,colframe=red!75!black}
\newtcolorbox{bluebox}{colback=blue!5!white,colframe=blue!75!black}
\newtcolorbox{greenbox}{colback=green!5!white,colframe=green!75!black}


\begin{document}

\title{Database Systems Query Costing}
\author{by Chris Chamberlain, current student - no guarantee of correctness}

\maketitle

\section{With an index}

\begin{greenbox}

\subsubsection*{Clustered index, matching one or more predicates.}

\begin{align*}
    \text{Cost(B+ tree index)} &= (\NPages(I) + \NPages(R)) \cdot \Pi_{i=1..n} RF_i \\
    \text{Cost(Hash index)} &= \NPages(R) \cdot \Pi_{i=1..n} RF_i \cdot 2.2
\end{align*}

\subsubsection*{Non-clustered index, matching one or more predicates.}

\begin{align*}
\text{Cost(B+ tree index)} &= (\NPages(I) + \NTuples(R)) \cdot \Pi_{i=1..n} RF_i \\
\text{Cost(Hash  index)} &= \NTuples(R) \cdot \Pi_{i=1..n} RF_i \cdot 2.2
\end{align*}

\subsubsection*{Only selecting a single tuple (index selection by primary key)}

\begin{align*}
\text{Cost(B+ tree)} &= \textrm{Height}(I) + 1 \\
\;\\
\text{Cost(Hash Probe)} &= \ProbeCost(I) + 1 \\
    &\approx 2.2 \\
\end{align*}

\end{greenbox}

{\small\textbf{Note:} the $+1$ in these two formulas is for accessing the actual data page.}

\section{Without using an index}

\begin{mybox}
\subsubsection*{Sequential heap scan (no index, unsorted):}
$$
\text{Cost} = \NPages(R)
$$

\textbf{Note:} heap scan is always an option and sometimes will be faster than using an index. Don't assume an index will be faster just because it exists.
\end{mybox}

\subsubsection*{Binary search (no index, sorted by relevant column):}
$$
\text{Cost} = \underbrace{\log_2(\NPages(R))}_{\text{\tiny Binary search cost to find first page}} + \overbrace{RF\cdot \NPages(R)}^{\text{\tiny Read all pages matching predicate}}
$$

\newpage
\section{Multi-relation plans / joins}

\begin{enumerate}
    \item Select order of relations (prefer ``left-deep-joins", allows for ``on-the-fly" reading and pipelining). Get rid of plans with cross-products
    \item For each join, select join algorithm (hash join, sort merge)
    \item For each input relation, select access method
\end{enumerate}

\subsection{Join algorithms}

$R$ is the outer relation, $S$ is the inner relation.

\begin{bluebox}
\begin{align*}
    \text{Cost(Tuple-oriented NLJ)} &= \NPages(R) + \NTuples(R) \cdot \NPages(S) \\
    \text{Cost(Page-oriented NLJ)} &= \NPages(R) + \NPages(R) \cdot \NPages(S) \\
    \text{Cost(Block-oriented NLJ)} &= \NPages(R) + \ceil*{\frac{\NPages(R)}{B - 2}} \cdot \NPages(S) \\
        &\text{where $B$ is the number of available memory pages} \\
        \; \\
    \text{Cost(Hash Join)} &= \underbrace{2 \cdot (\NPages(R) + \NPages(S))}_{\tiny \text{Partioning phase (read-write)}} + \underbrace{(\NPages(R) + \NPages(S))}_{\tiny \text{Matching phase (read)} }  \\
      &= 3 \cdot (\NPages(R) + \NPages(S))  \\
        \; \\
    \text{Cost(Sort-Merge Join)} &= \underbrace{2\cdot \NPages(R) \cdot \mathrm{NumPasses(R)}}_{\text{Sort outer}}\\
        &+ \underbrace{2 \cdot \NPages(S) \cdot \mathrm{NumPasses(S)}}_{\text{Sort inner}} \\
        &+ \underbrace{\NPages(R) + \NPages(S)}_{\text{Merge inputs}}
\end{align*}
\end{bluebox}

\subsection{Pipelining}

Todo.

\section{Cost of projections}

When performing a projection, you need to remove duplicates (\mintinline{SQL}{SELECT DISTINCT} in MySQL). There are two main methods:

\textbf{Sort-based projection: } (1) read table, keeping only projected attrs, (2) write pages with projected attrs to disk, (3) sort pages with external sort, (4) read sorted projected pages to discard adjacent duplicates.

\begin{greenbox}
\begin{align*}
\text{Cost(Sort-based projection)} = &\underbrace{\NPages(R)}_{\tiny \text{ReadTable}} + \underbrace{\NPages(R)\cdot PF}_{\tiny \text{WriteProjectedPages}} +\\
& \underbrace{2\cdot \text{NumPasses} \cdot \NPages(R) \cdot PF}_{ \text{SortProjectedPages}} + \underbrace{ \NPages(R) \cdot PF }_{ \text{ReadProjectedPages}} \\
\end{align*}
\end{greenbox}

\textbf{Hash-based projection: } (1) read table, keeping only projected attrs, (2) write pages with projected attrs into corresponding buckets, (3) read buckets one-by-one, create another hash-table and discard duplicates within a bucket.

\begin{greenbox}
\begin{align*}
\text{Cost(Hash-based projection)} = &\underbrace{\NPages(R)}_{\tiny \text{ReadTable}} + \underbrace{\NPages(R)\cdot PF}_{\text{WriteProjectedPages}} + \underbrace{ \NPages(R) \cdot PF }_{ \text{ReadProjectedPages}}
\end{align*}
\end{greenbox}

\newpage

\section{Appendix}

These formulas should be all mostly all intuitive, but they also underpin most of the formulas on the previous two pages.

\subsection{Reduction factors}

\begin{enumerate}

\item $col = value$

$$
RF = \frac{1}{\NKeys(col)}
$$

\item $col > value$

$$
RF = \frac{\High(col) - value}{\High(col) - \Low(col)}
$$

\item $col < value$

$$
RF = (val - \Low(col)) / (\High(col) - \Low(col))
$$

\item $col\_a = col\_b$ (join on a single column)

$$
RF = \frac{1}{\max(\NKeys(col\_a), \NKeys(col\_b))}
$$

\item No information, use a magic number $RF = \frac{1}{10}$

\end{enumerate}

\subsection{Result size after selection}

\textbf{Single table: } $\text{ResultSize} = \NTuples(R) \cdot \Pi_{i=1..n} \, RF_i$

\textbf{Joins:} $\text{ResultSize} = \Pi_{j=1..k} \NTuples(R_j) \cdot \Pi_{i=1..n} \, RF_i$

If there are no predicates, reduction factors are ignored (i.e.: they are all 1)


\end{document}